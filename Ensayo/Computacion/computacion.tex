\documentclass[15pt]{article}
\usepackage[spanish]{babel}
\usepackage{amsmath}
\usepackage{graphicx}

\begin{document}

\begin{center}

\bf{\sc\Huge Problemas que derivaron en la computacion moderna }\\
\end{center}

Por : Aly Cossio Caicedo
\large

\vspace{15PT}
Existen diversas cosas que históricamente han aportado a que la computación sea la que hoy en día todos conocemos, muchas de ellas han sido mas bien problemas, dudas o cuestionamientos, que en busca de soluciones llevó las mentes pensantes de aquella época, realizaran diseños (en su mayoría no probados) que a futuro servirían como bases para el desarrollo de sistemas computacionales modernos que usamos en la actualidad.

\vspace{15PT}
Todo se remonta a finales siglo XIX y principios del XX, cuando las matemáticas estaban pasando por un momento difícil, que es conocido al día de hoy como “crisis de los fundamentos”,teniendo asi impedimentoa para consolidar  terminos como  "infinito" en ella ,  salieron a la luz principalmente muchas inconsistencias o mejor dicho, paradojas, en muchos teoremas matemáticos; una paradoja muy conocida es la paradoja de cantor, que expone una verdad no absoluta sobre la teoría de los conjuntos.

\vspace{15PT}
Kurt Godel trabajo y demostró un teorema, llamado “teorema de la incompletitud”, mostrando asi que la teoría del matemático David Hilbert no era correcta, o por lo menos no del todo. El teorema que planteo Hilbert, expresa , expresa la oración Fq(q ) es indecidible en N , es decir en N no se puede deducir Fq(q ) ni su negación. Literalmente: :

"La analogía de esta argumentación con la antinomia de Richard salta a la vista; también está íntimamente relacionada con la. paradoja del mentiroso. pues la oración indecidible Fq(q ) dice que q pertenece a K, es decir según (7). que Fq(q ) no es deducible. Así pues, tenemos ante nosotros una oración que afirma su propia indeducibilidad. Evidentemente el método de prueba que acabamos de exponer es aplicable a cualquier sistema formal que, en primer lugar, interpretado naturalmente, disponga de medios de expresión suficientes para definir los conceptos que aparecen en la argumentación anterior especialmente el concepto de “fórmula deducible" y en el cual, en segundo lugar, cada. fórmula deducible sea verdadera en la interpretación natural."
(Cubo matematica ocupacional, 1999, p73) 

\vspace{15PT}
Alan Turing, es considerado pionero en cuanto a la informática y la computación se refiere, reconocido también por su gran influencia en la segunda guerra mundial, pues, este puso en jaque a los nazis al desencriptar todos sus mensajes, más específicamente los generados por la maquina enigma. Turing, también fue el creador del diseño de la reconocida máquina de Turing que fue quizás el la primer idea que abria las puertas a un futuro sistema computacional.
 
\vspace{15PT}
La máquina enigma era un sistema novedoso para la época, capaz de encriptar y desencriptar mensajes, que se usó de forma militar durante la segunda guerra mundial por los alemanes, desconociendo que esta maquina ya estaba siendo investigada por sus enemigos y en poco tiempo estos descubrieron la manera de desencriptar los mensajes (Alan Turing fue pieza clave de este descubrimiento); se dice que este suceso con la maquina enigma fue una de las razones de la pronta finalización de la segunda guerra mundial. 

\vspace{15PT}
Por otro lado, la maquina de Turing, fue un diseño cercano a lo que conocemos hoy, fue pensado como un sistema de lectura y escritura, que disponía de una cinta dividida en celdas las cuales podían o no contener información, ésta entraba en una parte especifica que era capaz de leer lo que decía la cinta y realizar la acción que allí se expresase (moverse a la derecha, izquierda, o escribir de ser necesario); pudiendo así, ser este un sistema automatizado es decir capaz de entender lo que se le expresa y realizar las acciones pertinentes. (cabe aclarar que el lenguaje que entendía la máquina, era simbólico, no literal.)

\vspace{15PT}
Todo lo anteriormente mencionado fue lo fundamental sobre los problemas que indirectamente generaron aportes a la computación, siendo todos estos diseños y conceptos rescatados en la actualidad y perfeccionados para crear los sistemas que todos conocemos y usamos ampliamente en la actualidad, siendo IBM un pionero en la creación de éstos.

\bibliography{https://www.researchgate.net/profile/Manuel_Alfonseca/publication/277189117_La_maquina_de_Turing/links/02e7e521b9154ce9ce000000.pdf}
\bibliography{https://www.nationalgeographic.com.es/ciencia/cuanto-sabes-sobre-alan-turing_14314/7}
\bibliography{https://books.google.es/books?hl=es&lr=&id=lgDGTYNcOY4C&oi=fnd&pg=PR13&dq=Godel&ots=SFgyLGN4Hf&sig=MX92eTcewPQ2ekD_y-Yw4sVR_uU#v=onepage&q=Godel&f=false}


\end{document}
